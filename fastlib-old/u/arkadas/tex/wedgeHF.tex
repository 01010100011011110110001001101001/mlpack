\documentclass[11pt]{article}

\begin{document}

\section{Introduction}

--Review Hartree-Fock in the wedge-product notation.

We will introduce a framework for calculating approximations to 
the ground state many-electron wavefunction for a given molecule, 
assuming the positions of the nuclei are given. This framework
reproduces the Hartree-Fock solution as the lowest order, and in
principle gives the exact wavefunction as the highest order
solution. The idea is to use some intermediate order approximation
that is computationally practical, and sufficiently accurate. 

We will start by reviewing the Hartree-Fock method in a new notation.

\subsection{Hartree-Fock theory}

The problem is to find an antisymmetric, $n$-electron wavefunction (actually, a spin orbital)
$\Psi =
\Psi(x_1, \cdots, x_n)$ that minimizes,
\begin{equation}
  E_\Psi = \langle \Psi | H | \Psi \rangle \,,
\end{equation}
where $H = H_{core} + H_{ee}$ is the Hamiltonian, with $H_{core}$
being the 1-electron core Hamiltonian representing electron kinetic energy
and the electron-nucleus interactions, and $H_{ee}$ representing the
electron-electron interactions,
\begin{equation}
  H_{ee} = \sum_{k=1}^{n-1} \sum_{l=k+1}^n \frac{e^2}{|\mathbf{r}_k - \mathbf{r}_l|}\,.
\end{equation}

The Hartree-Fock method restricts the search for the 
minimizing $\Psi$ to the set of Slater determinants of 1-electron spin
orbitals. This automatically gives $n$-electron wavefunctions that are
antisymmetric. Let us introduce a new notation to represent these
Slater determinants.

\subsection{Antisymmetric products}

Let $\psi_A(x)$ and $\psi_B(x)$ be two single-electron wavefunctions
(let us ignore spin for the moment; everything we say below can (and
in fact, should) be written in terms of spin orbitals). We define
their antisymmetric product, or the {\it wedge} product $\psi_A
\wedge \psi_B$ as an antisymmetric function for two variables,
\begin{equation}
  (\psi_A \wedge \psi_B)(x_1, x_2) = \frac{1}{\sqrt{2!}} \big(\psi_A(x_1)
  \psi_B(x_2) - \psi_A(x_2)\psi_B(x_1)\big)\,.
\end{equation}
Note that this implies $\psi_A\wedge \psi_B=-\psi_B\wedge\psi_A$. 
This is just a fancy notation for the Slater determinant of two
variables, but it will turn out to be convenient for the
generalization we will describe. As in the Slater determinant, the
factor $\frac{1}{\sqrt{2!}}$ ensures that the two-particle wavefunction
$\psi_A\wedge \psi_B$ is normalized when $\psi_A$ and $\psi_B$ are
orthogonal to each other\footnote{A note here on the fact that they
  can always be made orthogonal to each other without changing the
  determinant.} 
and are normalized as single-electron 
wavefunctions. Similarly, we can represent a
$3\times 3$ Slater determinant by a triple wedge product of
single-electron wavefunctions, which gives a function of three varibles,
\begin{eqnarray}
  (\psi_A \wedge \psi_B \wedge \psi_C)(x_1,x_2,x3) =
  \frac{1}{\sqrt{3!}}\Big[
  \psi_A(x_1) \psi_B(x_2) \psi_C(x_3) + \cr
  \psi_A(x_2) \psi_B(x_3) \psi_C(x_1) + \cr
  \psi_A(x_3) \psi_B(x_1) \psi_C(x_2) - \cr
  \psi_A(x_1) \psi_B(x_3) \psi_C(x_2) - \cr
  \psi_A(x_3) \psi_B(x_2) \psi_C(x_1) - \cr
  \psi_A(x_2) \psi_B(x_1) \psi_C(x_3)\Big] \,.
\end{eqnarray}
The $n$-fold product $\psi_1\wedge\cdots\wedge\psi_n$ is defined similarly,
by summing over all permutations of the arguments, and introducing the
appropriate signs just as in the definition of the determinant,
\begin{equation}
  (\psi_1\wedge\cdots\wedge\psi_n)(x_1,\cdots,x_n) = \frac{1}{\sqrt{n!}} \sum_{\sigma} 
  sgn(\sigma)\psi_1(x_{\sigma_1}) \psi_2(x_{\sigma_2})\cdots \psi_n(x_{\sigma_n})\,,
\end{equation}
where $\sigma$ denotes a permutation of $1\cdots n$, so the sum over
$\sigma$ denotes a sum over all such permutations. $sgn(\sigma)$
denotes the sign of the permutation, being equal to $1$ if $\sigma$
can be written as an even number of successive pair exchanges, and
$-1$ otherwise.

Although these definitions are enough for rewriting Hartree-Fock
theory using the notation of wedge products, we will also
define the wedge products of multiparticle wavefunctions, since these
will be crucial to the generalization we will describe.

Let $\omega$ be a two-particle, antisymmetric wavefunction, and $\psi($
be a single-particle wavefunction. We define their wedge product
$\omega \wedge \psi$ to be a totally antisymmetric three-particle
wavefunction given by,
\begin{eqnarray}
  (\omega \wedge \psi)(x_1, x_2, x_3) = \frac{1}{N_{2,1}} \Big[
\omega(x_1,x_2)\psi(x_3)+\cr
\omega(x_2,x_3)\psi(x_1)+\cr
\omega(x_3,x_1)\psi(x_2)-\cr
\omega(x_1,x_3)\psi(x_2)-\cr
\omega(x_3,x_2)\psi(x_1)-\cr
\omega(x_2,x_1)\psi(x_3)\Big] \,,
\end{eqnarray}
where $N_{2,1}$ is a normalization factor to be discussed shortly.
The idea of the definition is to permute the variables plugged in the three
``slots'' in $\omega(\cdot, \cdot)\psi(\cdot)$, and introduce the
appropriate signs for each permutation.\footnote{$\psi\wedge\omega$ is
  defined in a similar manner, and turns out to be equal to
  $\omega\wedge\psi$. Note that this commutative property does not
  hold in the more general case of a wedge product between a
  $p$-particle wavefunction and and a $q$-particle wavefunction.} 
Note that some of the terms
are identical to each other due to the antisymmetry of $\omega$, e.g.,
$-\omega(x_1,x_3)\psi(x_2) = \omega(x_3,x_1)\psi(x_2)$. The
normalization factor $N_{2,1} = \frac{1}{2\sqrt{3}}$ is 
chosen\footnote{The generalized version which will
  perhaps be discussed in a future version is given by
  $N_{p,q}=\left(p!q!{p+q\choose p}\right)^{-1}$} so that 
the wedge product $\omega\wedge\psi$ is
normalized when $\omega$ and
$\psi$ are normalized as two-particle and single-particle
wavefunctions, respectively, and are orthogonal to each other in the
sense that,
\begin{equation}
  \int \omega(x,y)\psi(y)dy = 0\,.
\end{equation}
Due to the antisymmetry of $\omega$, this implies $\int
\omega(y,x)\psi(y)dy=0$ also. We will come back to this notion of
orthogonality in the following sections. {\it Remark on the
  overcounting due to the antisymmetry of $\omega$}

This definition of the wedge product of a two-particle
wavefunction with a one-particle wavefunction is compatible with the
triple wedge product of single particle wavefunctions introduced above.
% give equation number
If one defines $\omega = \psi_A\wedge\psi_B$, then
$\omega \wedge \psi_C = (\psi_A\wedge\psi_B)\wedge\psi_C$
turns out to be equal to $\psi_A\wedge\psi_B\wedge\psi_C$ as given above.
% give equation number
Likewise, defining $\omega =
\psi_B\wedge\psi_C$ and using a definition of
$\psi\wedge\omega$ similar to that of $\omega\wedge\psi$, 
% give eq no
one can see that $\psi_A\wedge(\psi_B\wedge\psi_C) =
\psi_A\wedge\psi_B\wedge\psi_C$.

It is possible to generalize these definitions and 

--Orthogonality in one pair of slots implies that in any other pair.

--Remark on the fact that the orthogonalization of a two-part wavefn
  with a one-part. wavefne is always possible without changing their wedge
  product. (Generalize this to p-part and q-part.)

--Include remarks about the normalization factors not being standard.

\subsection{Orthogonality}

\subsection{The setup}

--Here, describe the general setup: we basically have a partially
ordered set of spaces of wavefunctions, the maximal element being the
set of all antisym n-particle wavefunctions, and the minimal one being
the set of Hartree-Fock wavefunctions.


\section{The first improvement}

Let us start by exploring the lowest order improvement over
Hartree-Fock theory in this framework, namely, minimizing the expectation
value of energy over the space of wavefunctions of the form,
\begin{equation}
  \Psi(x_1\cdots x_n) =
  (\omega\wedge\psi_3\wedge\psi_4\cdots\wedge\psi_n)(x_1\cdots x_n) \,,
\end{equation}
where $\omega$ is a normalized, two-particle, antisymmetric wavefunction,
$\{\psi_i: i=3\cdots n\}$ is an orthonormal set of single-particle
wavefunctions\footnote{Indexing of the $\psi$s starts from 3 for
 notational convenience.}, 
and $\omega$ is orthogonal to the $\psi_i$s in the sense
described above. % give section number
\subsection{Core terms}

Let's first calculate the expectation value of $H_{core}$ for this
wavefunction. Using $H_{core}(x_1,\cdots, x_n)=\sum_k
H_{core}^{(k)}(x_k)$,
\begin{equation}
  E_\Psi = \langle\Psi|H_{core}|\Psi\rangle = \sum_k
  \langle\Psi|H_{core}^{(k)}|\Psi\rangle\,,
\end{equation}
and expanding the wavefunction as a sum of
products with permuted arguments, we get,
\begin{eqnarray*}
 \lefteqn{E_\Psi=\int dx_1\cdots dx_n
  \sum_{k,\sigma,\tau}(-)^\sigma (-)^\tau}\\& & 
 \Big[\omega(x_{\sigma_1},x_{\sigma_2})
  \psi_3(x_{\sigma_3})\cdots\psi_n(x_{\sigma_n})\Big]H_{core}^{(k)}\Big[\omega(x_{\tau_1},x_{\tau_2})
  \psi_3(x_{\tau_3})\cdots\psi_n(x_{\tau_n})\Big]\,,
\end{eqnarray*}
where $\sigma$ and $\tau$ are permutations of $1\cdots n$, and
$(-)^\sigma$ and $(-)^\tau$ are shorthand for $sgn(\sigma)$ and
$sgn(\tau)$, respectively.

\subsubsection{Utilizing orthogonality}

\subsection{e-e terms}

Evaluate the e-e terms. 

\subsection{Total energy}

Give the final formula.

\section{Introducing a basis}

Extract the analogue of Roothaan-Hall equations by defining a basis. 

\section{Computational considerations}

How to implement a preliminary version of the algorithm.

\end{document}
