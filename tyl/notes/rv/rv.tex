\documentclass{article}
\usepackage{fullpage}
\title{A Syntax and Semantics for\\Probability and Statistics with Random Variables}
\author{Sooraj Bhat}

\parskip 0.7pc
\parindent 0pc

\newcommand{\E}[1]{\ensuremath{#1}}
\newcommand{\T}[1]{\texttt{#1}}
\newcommand{\rbool}{\T{bool}}
\newcommand{\rint}{\T{int}}
\newcommand{\rreal}{\T{real}}
\newcommand{\true}{\T{true}}
\newcommand{\false}{\T{false}}
\newcommand{\rlet}[3]{\E{\T{let}\ #1 \sim #2 \ \T{in}\ #3}}
\newcommand{\pdf}[2]{\E{\T{pdf(} #1 \ | \ #2 \T{)}}}
\newcommand{\prob}[2]{\E{\T{prob(} #1 \ | \ #2 \T{)}}}
\newcommand{\thereis}[3]{\E{\exists\ #1 : #2 \ . \ #3}}
\newcommand{\Not}{\T{not}}
\newcommand{\argmax}[2]{\E{\T{argmax}_{#1} #2}}
\newcommand{\case}{\T{case}}

\begin{document}
\maketitle
\section{Introduction}
%
why interesting, why unsolved, contributions.
%
\section{Syntax}
% 
\begin{eqnarray*}
t & ::= & \rbool \ | \ \rint \ | \ \rreal \\ %\ | \ {\tt pdf} \\
%\end{eqnarray*}
& & \\
%\begin{eqnarray*}
e & ::= & x \ | \ r 
    \ | \ -e \ | \ e_1 + e_2 \ | \ e_1 - e_2 
    \ | \ e_1 * e_2 \ | \ e_1 / e_2 \ | \ e_1^{e_2} \\
& & \ | \ \Not\ e \ | \ e_1 \vee e_2 \ | \ e_1 \wedge e_2 \\
& & \ | \ e_1 \leq e_2 \ | \ e_1 = e_2 \ | \ e_1 \geq e_2 \\
& & \ | \ \rlet{x}{e_1}{e_2} \\
& & \ | \ \prob{e_1}{e_2} 
\end{eqnarray*}

\rbool, \rint, and \rreal\ are synonyms for $[0,1]$,
$[-\infty,\infty]$, and $<-\infty,\infty>$. \true\ and \false\ are
synonyms for 0 and 1.  For now, we are disallowing arbitrary {\tt
let}s in the object language to solve the RV 'identity' problem.
Also, we disallow random variable literals (they would only be
marginally useful)...RVs can only be introduced with \T{rlet}.

Things to be added: \T{let}\ for multivariate/joint distributions,
\pdf{e_1}{e_2}, \thereis{x}{t}{e}, \argmax{x_1:t_1,\ldots}{e}, \case,
... indexing. \T{prob} is a special case of \T{pdf}.

Currently, no distinction is made between random reals and normal
reals, so that things like $f(x) = x^2 + 5$ operate on both.

\section{Examples}
%
\[ \rlet{x,y,z}{\mathrm{Normal}(0,1),\mathrm{Poisson}(x),\mathrm{Bernoulli}(3/4)}
       {\prob{x+3\leq y \wedge \Not\ z}{x=y \ \vee \ z}} \]
%
\[ \rlet{x,y,z}{\cdot,\cdot,\pdf{x}{y}}
   {\prob{x=3 \wedge y\leq\prob{x=2}{y=x}}{z}} \]
%
\[ \thereis{\mu,\sigma}{\rreal}{\rlet{x}{N(\mu,\sigma)}{\prob{x\leq x^2}{}}} \]
%
\[ \argmax{\mu_j,\sigma_j:\rreal}{ %
  \rlet{c_i,x_i}{\mathrm{Bernoulli}(1/2),\mathrm{Normal}(\mu_{c_i},\sigma_{c_i})}
       {\pdf{\{x_i=d_i\}}{}}
} \]

\begin{verbatim}
-- j in [0..1], i in [1..n]
exists mu[j], sigma[j] : real in
rlet c[i] ~ Bernoulli(1/2) in
rlet x[i] ~ Normal(mu[c[i]],sigma[c[i]]) in
argmax_{mu',sigma'} pdf( {x[i]=d[i]} | mu=mu' /\ sigma=sigma' )
\end{verbatim}

\section{Type System}

\rint\ and \rbool\ are refinements on \rreal.

type checking for the arithmetic operations are on the coarse type, \rreal.  


\section{Semantics}

\section{Things to support}

\begin{itemize}
\item probability
\item expectation
\item variance 
\item joint $\rightarrow$ marginalize
\item joint 
\item inequalities
\item goodness-of-fit, normality tests
\item order statistics
\item bounds , hoeffding's bound , specific dist. $\rightarrow$ special bounds
\item confidence intervals, hypothesis testing
\item likelihood, posterior, prior, evidence
\item convergence
\item sampling
\item mean, median, mode
\end{itemize}

\section{Related Work}

\begin{itemize}
\item PFP. Erwig.
\item APPL. Diane Evans.
\item Probability monads.
\item CES. Thrun.
\item Tyl. Agarwal.
\end{itemize}


\section{References}
\end{document}
