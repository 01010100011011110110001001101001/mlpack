\documentclass[twoside,leqno, 12pt]{article}
\usepackage{ltexpprt}

%\documentstyle[nips07submit_09,times]{article}

\usepackage{amsmath,amssymb}
\usepackage{graphicx}
\DeclareGraphicsRule{.tif}{png}{.png}{`convert #1 `dirname #1`/`basename #1 .tif`.png}

\newcommand{\spcA}{\hspace*{0in}}
\newcommand{\spcB}{\hspace*{.1in}}
\newcommand{\spcC}{\hspace*{.2in}}
\newcommand{\spcD}{\hspace*{.3in}}
\newcommand{\spcE}{\hspace*{.4in}}
\newcommand{\spcF}{\hspace*{.5in}}
\newcommand{\spcG}{\hspace*{.6in}}


%\title{NSF Research Proposal}
%\author{Bill March}
\date{}                                       

\begin{document}
%\maketitle
\begin{center}
\LARGE{NSF Graduate Research Fellowship Personal Essay} \newline
\Large{Bill March}
\end{center}

%NSF Fellows are expected to become knowledge experts and leaders who can contribute significantly to research, education, and innovations in science and engineering. The purpose of this essay is to demonstrate your potential to satisfy this requirement. Your ideas and examples do not have to be confined necessarily to the discipline that you have chosen to pursue. 

%Describe any personal, professional, or educational experiences or situations that have prepared you or contributed to your desire to pursue advanced study in science, technology, engineering, or mathematics. Describe your competencies and evidence of leadership potential. Discuss your career aspirations and how the NSF fellowship will enable you to achieve your goals. 

%%%%%%%%%% Outline: %%%%%%%%%
%
%Opening - write something about motivation and goals
%	Talk about wanting to work in science, especially chemistry and biology

%Experiences
%	Working with FASTlab
%	Teaching
%	Fraternity president
%	
%Goals

%NSF fellowship's role

%%%%%%%%

%%%% Still have about 1/2 page to work with, more about the fraternity?

%%% Opening #1
I see graduate school as the biggest opportunity of my life.  I have always read widely.  I always want to learn more, often about any topic.  I read widely in science and mathematics, but also history, literature, current events, anything I can get my hands on.  My home is filled with books, but there is one I always have prominently displayed: "The Bad Seed" by William March.  I don't particularly care for the book, and my name being the same as the author's is just a coincidence.  I always like seeing my own name among all the books.

Now that I am in grad school, that finally seems within reach.  I have a few classes and other obligations, but most of my time is now free to discover things.  Now I have the chance to write the book.  


% Not great, try to stay more focused on chemistry and biology from the beginning
Although my interests are very broad, I have always been drawn to science.  I chose to study mathematics instead, which lead into computers.  Throughout my studies, I was always searching for applications, which has lead me to computational chemistry.  




%%% TA experience
% Pretty good, should be able to use this in basically this form
% Do a bit more to drive home the points that: 1) I can communicate my findings to a broad audience, 2) interpret and communicate research findings
% Include that I got good reviews consistently, was asked to take on more courses
% This is about 1/2 page
\textbf{Teaching.}  For five semesters as an undergraduate, I worked as a teaching assistant in the School of Mathematics in courses on linear algebra, calculus, and differential equations.  I was responsible for grading tests and answering students questions.  However, the part of the job I really enjoyed was teaching the recitation sections.  Twice a week, I was responsible for leading a section of about 40 students for an hour.  The professors I worked for generally allowed me a lot of freedom to run the classes as I saw fit.  

I loved the challenge of communicating these complex ideas to (often disinterested) students.  I quickly came to appreciate the difficulty of this task.  In order for anything to stick, every concept had to be carefully motivated.  Complex details had to be carefully summarized.  Too many details, no matter how interesting I found them, quickly lost all of the students.  Too few, and they were unable to handle subtle problems later.  Through teaching, I found that I did not understand some of the concepts as well as I had thought, which motivated me to learn even more.  I found some of the best motivation came from examples in applications.  A student interested in computer graphics would doze through linear algebra until I motivated reflection of light as a matrix computation.  Realizing the power of this method motivated me to learn more about these applications.  I began bombarding friends in various fields for examples of eigenvectors and Laplace transforms in their classes.  My reading expanded even further, looking for more examples.  In the process, I came to appreciate even more the power of the things I was learning.  

%%% Experiences with FASTlab %%%

\textbf{FASTlab.}  In my last summer as an undergraduate, I started working with Alex Gray's research group.  I had my own research problems to work on, but they fit in the broad context of the group.  I quickly came to rely on the other graduate students to learn the code base and for reading suggestions.  I quickly found myself becoming a part of the larger group, despite still being an undergraduate.  

I enjoy working in a group environment.  I like discussing different research problems, interests, and views with the other students.  I often find suggestions from others helpful, especially since they have read differently and view problems differently.  

With the FASTlab, I have gained valuable experience in presenting my ideas to diverse audiences.  We have a weekly meeting where a student presents some open problems in his or her research.  I have presented on my work in computational geometry and computational biology and chemistry on several occasions.  Additionally, I have been involved in presentations to other research groups.  I was responsible for making slides and presenting in meetings with Jeffrey Skolnick and David Sherrill.  These presentations involved explaining difficult algorithmic ideas with roots in machine learning and statistics to audiences more familiar with biology and chemistry.  Additionally, in our current meetings with the chemistry people, I often find myself explaining concepts from computational chemistry to audiences more interested in machine learning

%% Stress that I'm translating in meetings with David and Alex

Both of these meetings have lead to collaborations between these groups and my lab, including my own proposal with Prof. Sherrill. 

%% Maybe try to add something about exploring my own ideas
% Definitely need to say something about the freedom to follow my own ideas and interests in chemistry.
% Want to say that the environment is supportive, currently pursuing some projects with people, etc.


\textbf{Goals.}

I intend to be a university professor after finishing my degree.  While there are many ways to pursue research, this method is most inline with my goals and interests.  It gives me the opportunity to continue teaching.  Also, working at a university will bring me into contact with researchers from many different scientific fields.  I will be able to develop collaborations with biologists and chemists interested in applying computation to problems in their fields.  I will have the opportunity to follow the computational trends in these fields.  

As a professor, I will also have the opportunity to continue teaching.  In addition to the challenge this provides, I think it is an excellent opportunity to further solidify new ideas by presenting them convincingly to more than just experts.  



%%%%%%%%
%%%%%%%%%%%%% Ideas %%%%%%%%%%%%%%%
%
%Grad school is a big opportunity.  Can read and think all the time, etc.  Don't talk about being young, it's cliche.  

%Long standing interest in science, course work drifted away but interest remained.  Scientists generally have to focus too narrowly on small systems.  Computing allows more big thinking.  Now have opportunity to bring experience to bear on science from an important angle: scientific computing.  

%I like a challenge - started college early, took hard classes, read all the time
%	Is any of this worthwhile, or is it just crap?

%Experiences + competencies/evidence of leadership potential:

%Other things I have done with FASTlab:
%	Helping other people with papers
%	Giving talks on my research
%	Discussing applications of my algorithm
%	Discussing varied research ideas
%	Presenting ideas to other labs: Skolnick, Sherrill
%	Interpreting in the discussions with chemistry people

%TA - teaching, education, communication - need to be clear
%	Lots of experience with explaining technical things to non-technical audience
%	Teaching linear algebra to uninterested freshmen who don't see why it's important
%	Relates to trying to communicate algorithms to scientists, science to algorithm people
%	Often found it easiest to interest students using applications they cared about
%		Graphics, electrical engineers, whatever
%	
%Fraternity president
%	leadership, setting goals, teamwork, organization
%	

%Career goals:

%Want to be a professor - can teach, work with many disciplines, etc.

%Want to make computing and mathematics accessible to scientists.  Requires ability to learn a lot of technical things, communicate scientific ideas and computational ideas back and forth.  
%Teaching is also an important part of this, since something. . .

%--
%Include something about NSF helping with all this
%	Having funding makes it easier to pursue collaborations with scientists
%	Can do my own work with many professors instead of having to fit in with one

%Why am I going to be able to do scientific computing?

%%%%%%%%%% Cover this stuff %%%%%%%%%%%%%%
%
%Intellectual Merit:

%Plan and conduct research - have done it before, will do it again
%	Came up with chemistry connections
%Work as a member of a team - working with the FASTlab, fraternity president
%Work independently - did my own algorithm
%Interpret and communicate research findings - talks, discussions with chemistry people

%
%Broader Impacts:

%Benefit society - computers are the way to find drugs, fight disease, study evolution, etc.
%Enhance scientific understanding - using computers
%Integrate research and education - have already given talks to my research group on computational biology, teaching others about QM, HF, AT, etc.
%	I like teaching, etc.
%Encourage Diversity - ???  I adopted an African child.  Everyone is made of atoms.  





\bibliographystyle{abbrv}
\bibliography{NSF_proposal}

\end{document}