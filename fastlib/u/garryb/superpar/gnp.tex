
\begin{document}

\begin{tabular}{ccc}
function              & name               & defined in terms of
\\
\hline
\sigma_q(Q')          & statistic          & \sigma_q'(Q'_L, Q'_R)
\\
\sigma_r(R')          & statistic          & \sigma_r(R'_L, R'_R)
\\
\hline
\delta(Q' \times R')  & delta              & 
\\
\mu(Q' \times R)      & mass result        & 
\\
\gamma(Q \times R)    & global stat        & 
\hline
\\
\phi(q \times r)      & pairwise           & 
\\
\lambda(q \times R')  & local accumulation & 
\\
\rho(q \times R)      & result per query   & 
\hline
\\
H(Q' \times R')       & heuristic          & 
\\
S(Q' \times R')       & prune              & 

\end{tabular}

\begin{verbatim}

I think I've characterized data dependence.  Neglecting the practical
considerations (such as difference between points, info, and bounds).

valid functions (note each occurence of 'f' is a distinct function, i'm lazy):

  f(Q) -> used for initial values for bounding info
  f(R) -> used for initial values of bounding info
  
  f(q)  -> point/info/transform     <- primitive
  f(Q') -> stat                     <- join( f(q) )

  f(r)  -> point/info/transform     <- primitive
  f(R') -> stat                     <- join( f(r) )
  
  f(Q' x R') -> delta               <- primitive
  f(Q' x R)  -> mass result         <- join( f(Q' x R'), f(q x R) )
  f(Q x R)   -> global stat         <- join( f(Q' x R'), f(q x r) )
  
  f(q x r)   -> pairwise evaluation <- primitive
  f(q x R')  -> local accumulator   <- join( f(q x r) )
  f(q x R)   -> result              <- join( f(q x R) )

some discourse on delta land:

  - delta land is interesting, it involves f(Q' x R') without any
  exploration
  - f(Q' x R') [delta] applies itself to f(Q' x R)
    - this change is uniform for all children
    - f(Q' x R') -> f(Q' x R)
    - if f(Q' x R') results in a prune, this does NOT need to be archived
    - if f(Q' x R') does not result in a prune, this must be archived to
    allow undoing

\end{verbatim}

\end{document}

